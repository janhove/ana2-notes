\documentclass[../main.tex]{subfiles}

\begin{document}
\chapter{Differentialrechnung}\label{chp:differential}
Dieses Kapitel stellt den Hauptteil dieser Vorlesung dar.
Auch in~\cite{heuser} ist das korrespondierende
Kapitel XX bei weitem das grösste. Der Inhalt ist das
Studium differenzierbarer Abbildungen 
$f \colon \mathbb{R}^m \to \mathbb{R}^n$.
Die erste Schwierigkeit daran ist, dass der Ausdruck
\[
  \lim_{h \to 0} \frac{f(p + h) - f(p)}{h}
\]
für $m \geq 2$ keinen Sinn macht: Vektoren kann man nicht
definieren.

\section{Differenzierbarkeit}
\begin{definition}
  Sei $U \subset \mathbb{R}^m$ offen. Eine Abbildung
  $f \colon U \to \mathbb{R}^n$ heisst
  \emph{differenzierbar} im Punkt $p \in U$, falls
  folgendes existiert:
  \begin{enumerate}[(i)]
    \item eine lineare Abbildung 
      ${(Df)}_p \colon \mathbb{R}^m \to \mathbb{R}^n$,
      genannt \emph{Differential} von $f$ an
      der Stelle $p$,
    \item für alle $h \in \mathbb{R}^m$ mit $p + h \in U$ 
      ein \emph{Restterm} ${(Rf)}_p(h) \in \mathbb{R}^n$,
      der relativ klein in $\Vert h \Vert_2$ ist,
      so dass für alle $h \in \mathbb{R}^m$ mit $p + h \in U$ 
      gilt, dass
      \[
        f(p+h) = f(p) + {(Df)}_p(h) + {(Rf)}_p(h).
      \]
  \end{enumerate}
  Der Unterteilung von $f$ in diese drei Summanden sagt man
  \emph{Dreigliedentwicklung}. Die Forderung,
  dass ${(Rf)}_p(h)$ \emph{relativ klein} in $\Vert h \Vert_2$ 
  ist, bedeutet, dass
  \[
    \lim_{h \to 0} \frac{\Vert {(Rf)}_p(h)\Vert_2}{\Vert h \Vert_2}
    = 0.
  \]
  In anderen Worten existiert für alle $\varepsilon > 0$ ein $\delta > 0 $,
  so dass für alle $h \in \mathbb{R}^m$ mit $\Vert h \Vert_2
  \leq \delta$ gilt, dass $\Vert {(Rf)}_p(h) \Vert_2
  \leq \varepsilon \cdot \Vert h \Vert_2$.
\end{definition}

\begin{example}
  Sei $L \colon \mathbb{R}^m \to \mathbb{R}^n$ linear.
  Dann gilt für alle $p, h \in \mathbb{R}^m$,
  dass
  \[
    L(p + h) = L(p) + L(h).
  \]
  Dies ist eine Dreigliedentwicklung für $L$ 
  an der Stelle $p$ mit Restterm ${(RL)}_p(h) = 0$.
  Tatsächlich ist die Abbildung ${(DL)}_p = L$ linear.
  Die informale Erklärung dafür ist,
  dass~$L$ die ``beste lineare Approximation'' von $L$ 
  ist, und das an jeder Stelle.
\end{example}

\begin{examples}
  \leavevmode
  \begin{enumerate}[(1)]
    \item Betrachte die Funktion
      \begin{align*}
        f \colon \mathbb{R}^m & \to \mathbb{R} \\
        p & \mapsto \langle p, p \rangle.
      \end{align*}
      Für alle $p, h \in \mathbb{R}^m$ gilt, dass
      \[
        f(p+ h) = \langle p + h, p + h \rangle
        = \langle p, p \rangle + 2 \langle p, h \rangle + \langle h, h \rangle.
      \]
      Dies ist eine Dreigliedentwicklung für $f$ mit
      Differential
      ${(Df)}_p(h) = 2 \langle p, h \rangle$ und
      Restterm
      ${(Rf)}_p(h) = \langle h, h \rangle$.
      Tatsächlich ist ${(Df)}_p$ linear in $h$,
      da Skalarprodukte bilinear sind, und
      \[
        {(Rf)}_p(h) = \langle h, h \rangle = \Vert h \Vert_2^2
      \]
      ist relativ klein in $\Vert h \Vert_2$.
      Zum Beweis dafür sei $\varepsilon > 0$ und setze
      $\delta = \varepsilon$.
      Für alle $h \in \mathbb{R}^m$ mit $\Vert h \Vert_2 \leq \delta$ 
      gilt dann, dass
      \(
        |{(Rf)}_p(h)| = \Vert h \Vert_2^2 \leq \varepsilon \cdot \Vert h \Vert_2
      \).
    \item Betrachte die Abbildung
      \begin{align*}
        m \colon \mathbb{R}^2 & \to \mathbb{R} \\
        (x, y) & \mapsto xy.
      \end{align*}
      Seien $p = (p_1, p_2)$ und $h = (h_1, h_2)$ in $\mathbb{R}^2$.
      Dann gilt
      \[
        m(p + h) = (p_1 + h_1) \cdot (p_2 + h_2)
        = p_1 p_2 + (p_1 h_2 + p_2 h_1) + h_1 h_2.
      \]
      Dies ist eine Dreigliedentwicklung für $m$ bei $p$
      mit ${(Dm)}_p(h) = p_1 h_2 + p_2 h_1$ und
      ${(Rm)}_p(h) = h_1 h_2$.
      Da
      \[
        {(Dm)}_p(h) =
        \begin{pmatrix}
          p_2 & p_1
        \end{pmatrix}
        \begin{pmatrix}
          h_1 \\ h_2
        \end{pmatrix}
      \]
      gilt, ist ${(Dm)}_p$ linear.
      Weiter ist der Restterm 
      relativ klein in $\Vert h \Vert_2$, da
      \[
        |{(Rm)}_p(h)| = |h_1| \cdot |h_2| \leq \Vert h \Vert_2 \cdot
        \Vert h \Vert_2 = \Vert h \Vert_2^2,
      \]
      und $\Vert h \Vert_2^2$ ist nach Beispiel (1)
      relativ klein in $\Vert h \Vert_2$.
  \end{enumerate}
\end{examples}

Wir treffen nun die Kettenregel als Quelle
für viele Beispiele an.
Wir werden sie später in diesem Kapitel beweisen.

\begin{theorem*}[Kettenregel]
  Seien $U \subset \mathbb{R}^m$ und $V \subset \mathbb{R}^k$ 
  offen, und sei $f \colon U \to \mathbb{R}^k$ differenzierbar
  bei $p \in U$ mit $f(U) \subset V$,
  sowie $g \colon V \to \mathbb{R}^n$ bei $f(p) \in V$ 
  differenzierbar.
  Dann ist die Komposition $g \circ f \colon U \to \mathbb{R}^n$ 
  differenzierbar bei $p$, und es gilt
  \[
    {(D(g \circ f))}_p = {(Dg)}_{f(p)} \circ {(Df)}_p.
  \]
\end{theorem*}

\begin{example}
  Betrachte die Abbildung
  \begin{align*}
    \ell \colon \mathbb{R}^m & \to \mathbb{R} \\
    p & \mapsto \sqrt{\langle p, p \rangle}.
  \end{align*}
  In Koordinaten $x_1, \dots, x_m$ ist
   \[
     \ell(x_1, \dots, x_m) = \sqrt{x_1^2 + \cdots + x_m^2}.
  \]
  Sei $U = \mathbb{R}^m \setminus \{0\} \subset \mathbb{R}^m$ 
  und $V = \mathbb{R}_{>0}$.
  Beide dieser Mengen sind offen. Setze weiterhin
  \begin{align*}
    f \colon U & \to \mathbb{R} \\
    p & \mapsto \langle p, p \rangle
  \end{align*}
  und
  \begin{align*}
    g \colon \mathbb{R}_{>0} & \to \mathbb{R} \\
    t & \mapsto \sqrt t.
  \end{align*}
  Für alle $p \in U$ gilt $\ell(p) = g(f(p))$.
  Die Funktion $f$ ist differenzierbar in allen
  Punkten $p \in U$ 
  (eigentlich sogar in allen $p \in \mathbb{R}^n$).
  Ausserdem ist $g$ differenzierbar an allen Stellen
  $t > 0$.
  Aus der Analysis I wissen wir, dass
  \[
    {(Dg)}_t(h) = g'(t) \cdot h = \frac{1}{2 \sqrt t} \cdot h.
  \]
  Die Kettenregel liefert, dass
  $\ell = g \circ f$ an jeder Stelle $p \in U$ differenzierbar ist,
  und es gilt
  \[
    {(D \ell)}_p(h) = {(Dg)}_{f(p)}({(Df)}_p(h))
    = \frac{1}{2 \sqrt{\langle p, p \rangle}} \cdot 2 \langle p, h \rangle
    = \langle p / \Vert p \Vert_2, h \rangle.
  \]
  Der Vektor $p / \Vert p \Vert_2$ ist der Vektor mit Länge $1$,
  der in die selbe Richtung zeigt wie $p$.
  Er beschreibt den ``Zuwachs'' der Funktion $\ell$.
\end{example}

\begin{definition}
  Sei $f \colon \mathbb{R}^m \to \mathbb{R}^n$ eine Abbildung.
  Dann existieren eindeutige \emph{Komponentenfunktionen}
  $f_k \colon R^m \to \mathbb{R}$ 
  für $1 \leq k \leq n$ mit folgender Eigenschaft:
  für alle $p \in \mathbb{R}^m$ gilt, dass
  \[
    f(p) = \sum_{k=1}^{n} f_k(p) \cdot e_k.
  \]
\end{definition}

\begin{lemma}
  Eine Abbildung $f \colon \mathbb{R}^m \to \mathbb{R}^n$ ist differenzierbar,
  genau dann wenn alle Komponentenfunktionen 
  $f_k \colon \mathbb{R}^m \to \mathbb{R}$ bei $p$ differenzierbar sind.
\end{lemma}

\begin{proof}
  Um zu zeigen, dass die Komponentenfunktionen einer
  differenzierbaren Abbildung $f \colon \mathbb{R}^m \to \mathbb{R}^n$
  selbst differenzierbar sind, sei
  \[
    f(p + h) = f(p) + {(Df)}_p(h) + {(Rf)}_p(h)
  \]
  eine Dreigliedentwicklung von $f$.
  Für $1 \leq k \leq n$ 
  setze ${(Df_k)}_p(h) = \langle {(Df)}_p(h), e_k \rangle$ 
  und ${(Rf_k)}_p(h) = \langle {(Rf)}_p(h), e_k \rangle$.
  Dann gilt für alle $k \leq n$, dass
  \[
    f_k(p + h) = f_k(p) + {(Df_k)}_p(h) + {(Rf_k)}_p(h).
  \]
  Weiterhin ist
  \begin{enumerate}[(i)]
    \item die Funktion
      \begin{align*}
        {(Df_k)}_p \colon \mathbb{R}^m & \to \mathbb{R} \\
        h & \mapsto \langle {(Df)}_p(h), e_k \rangle
      \end{align*}
      als Verknüpfung einer linearen Funktion mit einer Projektion
      linear,
    \item der Restterm ${(Rf_k)}_p(h)$ relativ klein
      in $\Vert h \Vert_2$, da 
      \(
        |{(Rf_k)}_p(h)| \leq \Vert {(Rf)}_p(h) \Vert_2
      \)
      gilt.
  \end{enumerate}

  Umgekehrt nehmen wir an, dass die Komponentenfunktionen
  $f_k \colon \mathbb{R}^m \to \mathbb{R}$
  einer Funktion $f \colon \mathbb{R}^m \to \mathbb{R}^n$ 
  differenzierbar sind.
  Für alle $k$ gibt es dann eine Dreigliedentwicklung
  \(
    f_k(p + h) = f_k(p) + {(Df_k)}_p(h) + {(Rf_k)}_p(h).
  \)
  Setze
  \[
    {(Df)}_p(h) = \sum_{k=1}^{n} {(Df_k)}_p(h) \cdot e_k
  \]
  und
  \[
    {(Rf)}_p(h) = \sum_{k=1}^{n} {(Rf_k)}_p(h) \cdot e_k.
  \]
  Dann gilt
  \(
    f(p + h) = f(p) + {(Df)}_p(h) + {(Rf)}_p(h).
  \)
  Weiterhin ist
  \begin{enumerate}[(i)]
    \item die Abbildung
      \begin{align*}
        {(Df)}_p \colon \mathbb{R}^m & \to \mathbb{R}^n \\
        h & \mapsto \sum_{k=1}^{n} {(Df_k)}_p(h) \cdot e_k
      \end{align*}
      linear da Summen linearer Abbildungen linear sind, und
    \item ${(Rf)}_p(h)$ ist relativ klein in $\Vert h \Vert_2$,
      da
      \[
        \Vert {(Rf)}_p(h) \Vert_2 \leq \sum_{k=1}^{n} |{(Rf_k)}_p(h)|.
      \]
      Dies folgt aus der allgemeinen Ungleichung, dass
      die euklidische Norm immer kleiner ist als die Summennorm.
      Weiterhin verwenden wir, dass endliche Summen
      relativ kleiner Funktionen selbst relativ klein ist.
      Dazu teile $\varepsilon$ durch $n$ um ein geeignetes
      $\delta$ als Minimum $n$ verschiedener $\delta$ zu bekommen.
      \qedhere
  \end{enumerate}
\end{proof}

\begin{question}
  Können wir die Differenzierbarkeit von Abbildungen
  $f \colon \mathbb{R}^m \to \mathbb{R}$ weiter 
  auf die Differenzierbarkeit von Funktionen
  $g \colon \mathbb{R} \to \mathbb{R}$ zurückführen?
  Konkreter: Ist $f \colon \mathbb{R}^m \to \mathbb{R}$ 
  bei $p = 0$ differenzierbar, falls die Einschränkung
  von $f$ auf allen Koordinatenachsen differenzierbar ist?
\end{question}

Die Antwort hier ist leider ``nein'', wie in folgendem
Beispiel zu erkennen ist.

\begin{example}
  Betrachte die Funktion
  \begin{align*}
    f \colon \mathbb{R}^2 & \to \mathbb{R} \\
    (x, y) & \mapsto 
    \begin{cases}
      \frac{xy}{x^2 + y^2} & (x, y) \neq (0, 0) \\
      0 & (x, y) = (0, 0).
    \end{cases}
  \end{align*}
  Die Einschränkung von $f$ auf beide Achsen ist identisch null,
  also differenzierbar. Aber $f$ ist nicht einmal stetig im Nullpunkt:
  \[
    \lim_{n \to \infty} f(1/n, 1/n) = \frac{1}{2} \neq f(0, 0).
  \]
  Vergleiche auch Serie 6 für ein Beispiel einer Funktion, die
  trotz Stetigkeit im Nullpunkt dort nicht differenzierbar ist.
\end{example}



\end{document}

