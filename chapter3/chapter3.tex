\documentclass[../main.tex]{subfiles}

\begin{document}
\chapter{Gradientenfelder und Differentialformen}\label{chp:gradients}
\section{Gradientenfelder}
Der Stoff dieses Abschnitts ist in den
Abschnitten 181 und 182 in~\cite{heuser} zu finden.

Das motivierende Problem ist folgendermassen zu formulieren.
Sei $U \subset \mathbb{R}^n$ offen und
$X \colon U \to \mathbb{R}^n$ ein stetiges Vektorfeld.
Ist $X$ ein \emph{Gradientenfeld}, das heisst,
existiert
eine differenzierbare Funktion
$f \colon U \to \mathbb{R}^n$,
so dass für alle $p \in U$ gilt, dass ${(\nabla f)}_p = X(p)$?
Dasselbe Problem lässt sich auch in Koordinaten stellen.
Schreibe
\[
  X(p) = \sum_{k=1}^{n} a_k(p) e_k
\]
mit stetigen Komponentenfunktionen $a_k \colon U \to \mathbb{R}$.
Gesucht ist $f \colon U \to \mathbb{R}$ mit
\[
  \frac{\partial f}{\partial x_k} (p) = a_k(p)
\]
für alle
$k \leq n$ und alle $p \in U$.

\begin{specialcase}
  Wir betrachten den Spezialfall $n = 1$.
  Sei $X \colon \mathbb{R} \to \mathbb{R}$ stetig.
  Definiere die Funktion
  $f \colon \mathbb{R} \to \mathbb{R}$ 
  durch 
  \[
    f(p) = \int_{0}^{p} X(t) \, dt.
  \]
  Dann gilt für alle $p \in \mathbb{R}$,
  dass $f'(p) = X(p)$.
  Auf $\mathbb{R}$ sind also alle stetigen Vektorfelder Gradientenfelder.
\end{specialcase}

In Dimension $n \geq 2$ sind nicht alle stetigen
Vektorfelder Gradientenfelder.
Wir werden dazu ein Beispiel konstruieren, indem wir
den Fall $n = 2$ genauer betrachten.
Betrachte ein Vektorfeld $X \colon \mathbb{R}^2 \to \mathbb{R}^2$
mit Komponentenfunktionen $a(x, y)$ und $b(x, y)$.
Sei $f \colon \mathbb{R}^2 \to \mathbb{R}$ differenzierbar
mit $\nabla f = X$.
Falls $b(x, y)$ konstant null ist (insbesondere auch
$\partial f / \partial y = 0)$, dann hängt
$f(x, y)$ nicht von $y$ ab.
Es existiert also eine
differenzierbare Funktion
$g \colon \mathbb{R} \to \mathbb{R}$,
so dass für alle $(x, y) \in \mathbb{R}^2$ gilt,
dass $f(x, y) = g(x)$.
Dann folgt $\partial f / \partial x (x, y) = g'(x)$.
Also hängt $a(x, y) = \partial f / \partial x(x y) = g'(x)$
nicht von $y$ ab.

\begin{example}
  Das Vektorfeld $X \colon \mathbb{R}^2 \to \mathbb{R}^2$ 
  mit $X(x, y) = (y, 0)$ ist kein Gradientenfeld,
  da $a(x, y) = y$ von $y$ abhängt.
\end{example}

Sei nun $U \subset \mathbb{R}^n$ offen und
\emph{wegzusammenhängend}, das heisst, für
alle $p , q \in U$ existiert eine stetig
differenzierbare
Kurve $\gamma \colon [0, 1] \to U$ mit $\gamma(0) = p$
und $\gamma(1) = q$.
Die Forderung, dass eine stetige Kurve $\gamma$
existiert, impliziert bereits, dass auch eine stetig
differenzierbare Kurve existiert. Wir werden darauf
aber nicht weiter eingehen.

\begin{proposition}\label{prop:gradient-fields-curves}
  Sei $U \subset \mathbb{R}^n$ offen
  und wegzusammenhängend und
  sei $X \colon U \to \mathbb{R}^n$ stetig.
  Dann existiert eine differenzierbare Funktion
  $f \colon U \to \mathbb{R}$ mit $\nabla f = X$,
  genau dann, wenn für alle stetig differenzierbaren
  Wege $\gamma, \overline \gamma \colon [0, 1] \to U$ 
  mit $\gamma(0) = \overline \gamma ( 0) $ und
  $\gamma(1) = \overline\gamma(1)$ gilt, dass
  \[
    \int_{0}^{1} \langle X(\gamma(t)), \dot \gamma(t) \rangle \, dt
    =
    \int_{0}^{1} \langle X(\overline\gamma(t)), 
    \dot{\overline\gamma}(t) \rangle \, dt.
  \]
\end{proposition}

In Worten hängt das Wegintegral von $X$ nur von den Endpunkten
der betrachteten Kurve ab. Äquivalenterweise ist das Integral
von $X$ über alle geschlossenen Wege null.
Geschlossen heisst hier $\gamma(0) = \gamma(1)$.
In diesem Fall gilt für $\overline \gamma(t) = \gamma(0)$ 
dann, dass
\[
  \int_{0}^{1} \langle X(\gamma(t)), \dot \gamma(t) \rangle \, dt
  =
  \int_{0}^{1} \langle X(\overline\gamma(t)), 
  \dot{\overline\gamma}(t) \rangle \, dt = 0.
\]
Umgekehrt seien $\gamma, \overline \gamma \colon [0, 1] \to U$ 
mit $\gamma(0) = \overline \gamma(0)$.
Dann bastle aus $\gamma$ und $\overline \gamma$ 
einen geschlossenen Weg, der zuerst
$\gamma$ durchläuft, und darauf $\overline \gamma$ in umgekehrter
Richtung durchläuft.
Präziser betrachten wir den Weg
\begin{align*}
  \gamma \cup - \overline \gamma \colon [0, 1] & \to U \\
  t & \mapsto 
  \begin{cases}
    \gamma(2t) & t \leq 1/2,\\
    \overline \gamma (2 - 2t) & t \geq 1/2.
  \end{cases}
\end{align*}


\begin{proof}[Beweis von Proposition~\ref{prop:gradient-fields-curves}]
  Sei $f \colon U \to \mathbb{R}$ differenzierbar
  mit $\nabla f = X$.
  Sei weiterhin $\gamma \colon [0, 1] \to U$ stetig differenzierbar.
  Für die Komposition $f \circ \gamma \colon [0, 1] \to \mathbb{R}$ 
  gilt dann für alle $t \in (0, 1)$, dass
  \[
    \frac{d}{dt}(f \circ \gamma) (t)
    = {(Df)}_{\gamma(t)}({(D\gamma)}_t(1))
    = \langle {(\nabla f)}_{\gamma(t)}, \dot \gamma(t) \rangle.
  \]
  Wir folgern mit Hilfe der stetigen Differenzierbarkeit
  von $\gamma$, dass
  \begin{align*}
    \int_{0}^{1} \langle X(\gamma(t)), \dot \gamma(t) \rangle \, dt
    &= \int_{0}^{1} \langle {(\nabla f)}_{\gamma(t)}, 
    \dot \gamma(t) \rangle \, dt \\
    &= \int_{0}^{1} \frac{d}{dt} (f \circ \gamma)(t) \, dt  \\
    &= f(\gamma(1)) - f(\gamma(0)).
  \end{align*}
  Insbesondere ist, falls $\gamma(1) = \gamma(0)$ gilt, das Wegintegral
  von $X$ null.

  Sei nun umgekehrt $X \colon U \to \mathbb{R}^n$
  ein stetig differenzierbares Vektorfeld, dessen geschlossenen
  Wegintegrale alle verschwinden.
  Wir konstruieren eine \emph{Potentialfunktion}
  $f \colon U \to \mathbb{R}$ mit $\nabla f = X$.
  Sei $p \in U $ fest gewählt.
  Sei $q \in U$.
  Wähle $\gamma \colon [0, 1] \to U$ stetig differenzierbar
  mit $\gamma(0) = p$ und $\gamma(1) = q$.
  Setze
   \[
     f(q) = \int_{0}^{1} \langle X(\gamma(t)), \dot \gamma(t) \rangle \, dt.
  \]
  Wir bemerken, dass $f(q)$ nicht von der Wahl von $\gamma$ abhängt.
  Wir zeigen nun, dass für alle $q \in U$ gilt,
  dass ${(\nabla f)}_q = X(q)$.
  Sei dazu $h \in \mathbb{R}^n$ beliebig.
  Für kleine $t \in \mathbb{R}$ gilt dann, dass $q + th$
  auch noch in $U$ liegt (denn $U$ ist offen).
  Wir haben
  \[
    \langle {(\nabla f)}_q, h \rangle
    = {(Df)}_q(h) = \lim_{t \to 0} \frac{f(q + th) - f(q)}{t}.
  \]
  Betrachte den Weg  
  \begin{align*}
    \delta \colon [0, t] & \to U \\
    s & \mapsto q + sh.
  \end{align*}
  Es existiert $t > 0$ so, dass $\delta([0, t]) \subset U$.
  Nach Konstruktion gilt
  \begin{align*}
     f(q + th) - f(q) 
     &=\int_{0}^{t} \langle X(\delta(s)), \dot \delta(s) \rangle \, ds  \\
     &= \int_{0}^{t} \langle X(q + sh), h \rangle \, ds \\
     &= \int_{0}^{t} \langle X(q), h \rangle \, ds
     + \int_{0}^{t} \langle X(q + sh) - X(q), h \rangle \, ds \\
     &= t \langle X(q), h \rangle + R
  \end{align*}
  für
  \[
    R = \int_{0}^{t} \langle X(q + sh) - X(q), h \rangle \, ds.
  \]
  Schätze ab, dass
  \[
    |R| \leq t \cdot \Vert X (q + sh) - X(q) \Vert_2 \cdot \Vert h \Vert_2.
  \]
  Es gilt also
  \[
    \lim_{t \to 0} \frac{|R|}{t} = 0,
  \]
  da $X$ stetig an der Stelle $q$ ist.
  Daraus folgt, dass
  \[
    {(Df)}_q (h) = \lim_{t \to 0} \frac{f(q + th) - f(q)}{t}
    = \langle X(q), h \rangle,
  \]
  was ${(\nabla f)}_q = X(q)$ impliziert.
\end{proof}

\begin{example}
  Sei
  \begin{align*}
    X \colon \mathbb{R}^2 & \to \mathbb{R}^2 \\
    (x, y) & \mapsto (-y, x)
  \end{align*}
  eine Rotation um den Winkel $\pi/2$.
  Betrachte den Weg
  \begin{align*}
    \gamma \colon [0, 1] & \to \mathbb{R}^2 \\
    t & \mapsto (\cos(2 \pi t), \sin(2 \pi t)).
  \end{align*}
  Es gilt $\gamma(0) = \gamma(1) = (1, 0)$.
  Berechne aber
  \[
    \int_{0}^{1} \langle X(\gamma(t)), \dot \gamma(t) \rangle \, dt
    = 2\pi \cdot \int_{0}^{1} \cos^2(2 \pi t) + \sin^2(2 \pi t) \, dt
    = 2 \pi \neq 0.
  \]
  Somit ist $X$ kein Gradientenfeld.
\end{example}

\subsection*{Lemma von Poincaré}

Leider ist die Bedingung, dass alle geschlossenen Wegintegrale
null sind, impraktikabel.
Die Bedingung kann nämlich nur schwer dazu verwendet werden,
zu verifizieren, dass eine gegebenes Vektorfeld $X$ 
ein Gradientenfeld ist.
Das Lemma von Poincaré unten liefert ein leichter überprüfbares
Kriterium

Sei $U \subset \mathbb{R}^n$ offen
und $X \colon U \to \mathbb{R}^n$ differenzierbar.
Falls eine differenzierbare Funktion
$f \colon U \to \mathbb{R}$ mit $\nabla f = X$ existiert,
dann ist $f$ automatisch zweimal differenzierbar
(da $\nabla f = X \colon U \to \mathbb{R}^n$ differenzierbar ist).
Nach dem Satz von Schwarz gilt für alle
$i, j \in \{1, \dots, n\}$ und alle $p \in U$, dass
\[
  \frac{\partial^2 f}{\partial x_j \partial x_i}(p) =
  \frac{\partial^2 f}{\partial x_i \partial x_j}(p).
\]
Schreibe nun
\[
  X(p) = \sum_{k=1}^{n} a_k(p) e_k
\]
mit differenzierbaren Komponentenfunktionen $a_k \colon U \to \mathbb{R}$.
Aus $\nabla f = X$ folgen nun die ``Poincaré-Bedingungen''
\[
  \frac{\partial a_i}{\partial x_j}(p) = \frac{\partial a_j}{\partial x_i}(p).
\]

Wir betrachten nun den Spezialfall $n = 3$.
Sei $X \colon \mathbb{R}^3 \to \mathbb{R}^3$ 
differenzierbar mit Komponenten $a, b, c \colon \mathbb{R}^3 \to \mathbb{R}$.
Dann sind die Poincaré-Bedingungen äquivalent zu den Gleichungen
\begin{align*}
  \frac{\partial a}{\partial y} &= \frac{\partial b}{\partial x}, \\
  \frac{\partial a}{\partial z} &= \frac{\partial c}{\partial x}, \\
  \frac{\partial b}{\partial z} &= \frac{\partial c}{\partial y}.
\end{align*}

\begin{definition}
  Sei $X \colon \mathbb{R}^3 \to \mathbb{R}^3$ differenzierbar
  mit Komponenten $a, b, c$.
  Dann ist die \emph{Rotation} von $X$ gegeben durch
  \[
    \text{rot}(X) =
    \begin{pmatrix}
      {\partial b}/{\partial z} - {\partial c}/{\partial y} \\
      {\partial c}/{\partial x} - {\partial a}/{\partial z} \\
      {\partial a}/{\partial y} - {\partial b}/{\partial x}.
    \end{pmatrix}
  \]
\end{definition}

Es folgt direkt, dass die Poincaré-Bedingungen für
ein differenzierbares Vektorfeld $X$ äquivalent
zur Forderung $\text{rot}(X) = 0$ sind.
Leider reichen die Poincaré-Bedingungen im allgemeinen
nicht aus, um ein Vektorfeld $X$ als Gradientenfeld
zu charakterisieren.

\begin{example}
  Sei $U = \mathbb{R}^2 \setminus \{0\}$.
  Sei  
  \begin{align*}
    X \colon \mathbb{R}^2 \setminus \{0\} & \to \mathbb{R}^2 \\
    (x, y) & \mapsto (-y/(x^2 + y^2), x/(x^2 + y^2)).
  \end{align*}
  Es gilt für alle $x, y \neq (0, 0)$, dass
  $\partial a / \partial y (x, y) = \partial b/ \partial x(x, y)$,
  vergleiche Serie 10.
  Wir bemerken aber, dass die Einschränkung von $X$ 
  auf $S^1 = \left\{(x, y) \in \mathbb{R}^2 \mid x^2 + y^2 = 1\right\}$ 
  mit dem Vektorfeld $\overline X(x, y) = (-y, x)$ übereinstimmt.
  Vom Vektorfeld $\overline X$ wissen wir aus einem Beispiel
  oben aber bereits, dass es ein geschlossenes nicht-verschwindendes
  Wegintegral zulässt.
  Konkreter gilt für den Weg
  $\gamma(t) = (\cos(2 \pi t), \sin (2 \pi t))$, dass
  \[
    \int_{0}^{1} \langle X(\gamma(t)), \dot \gamma(t) \rangle \, dt
    = 2\pi,
  \]
  also ist $X$ kein Gradientenfeld.
  
\end{example}





\end{document}

